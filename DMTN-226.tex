\documentclass[DM,authoryear,toc]{lsstdoc}
% lsstdoc documentation: https://lsst-texmf.lsst.io/lsstdoc.html
\input{meta}

% Package imports go here.

% Local commands go here.

%If you want glossaries
%\input{aglossary.tex}
%\makeglossaries

\title{Rubin/LSST Alert Filtering System}

% Optional subtitle
% \setDocSubtitle{A subtitle}

\author{%
Leanne Guy, Eric C. Bellm
}

\setDocRef{DMTN-226}
\setDocUpstreamLocation{\url{https://github.com/lsst-dm/dmtn-226}}

\date{\vcsDate}

% Optional: name of the document's curator
% \setDocCurator{The Curator of this Document}

\setDocAbstract{%
This document presents a strategy for using the ANTARES alert broker as the Rubin Observatory Alert Filtering System.
}

% Change history defined here.
% Order: oldest first.
% Fields: VERSION, DATE, DESCRIPTION, OWNER NAME.
% See LPM-51 for version number policy.
\setDocChangeRecord{%
  \addtohist{1}{YYYY-MM-DD}{Unreleased.}{Eric Bellm}
}


\begin{document}

% Create the title page.
\maketitle
% Frequently for a technote we do not want a title page  uncomment this to remove the title page and changelog.
% use \mkshorttitle to remove the extra pages

% ADD CONTENT HERE
% You can also use the \input command to include several content files.

\section{Background}

During steady-state operations, Rubin Observatory's LSST will produce about ten million world-public alerts of transients, variable, and moving objects.
Science users will work with community alert brokers to crossmatch, filter, and classify these alerts in order to identify the subset that require real-time followup observations.
At the time the vision for this ambitious and comprehensive alert system was developed, no community alert brokers were functional.
Accordingly, the Rubin project provided a fallback: a simple ``alert filtering system'' (AFS; \S \ref{sec:requirements}) that the project would provide.
The AFS would allow Rubin data-rights holders to apply simple user-defined or pre-defined filters to the Rubin alert stream.
This minimized the risk that community alert brokers would not be available, or not provide the functionality required to fulfill users' needs.

Today, the sitution is quite different.
Seven community alert brokers\footnote{\url{https://www.lsst.org/scientists/alert-brokers}} have been approved for direct access to the full Rubin alert stream, with two more planning to operate downstream of a full-stream broker.
The brokers have mature systems and active user bases and are operating on live Zwicky Transient Facility alerts.
Several brokers have functionality which mirrors that planned for the Rubin AFS.
Accordingly, the rationale for a project-provided AFS is less clear.

However, it is reasonable for the Rubin project to ensure that Rubin data-rights holders have access to AFS-like capabilities throughout the survey, even if that service is provided by another entity.
In this technote we propose a partnership with the ANTARES alert broker \citep{2021AJ....161..107M}, one of the seven approved full-stream brokers. 
Like the Rubin Operations project, ANTARES operates as part of NOIRLab.

In \S \ref{sec:requirements} we review the formal requirements for the Rubin AFS.
\S \ref{sec:antares} describes the ANTARES alert broker.
In \S \ref{sec:implementation} we discuss steps necessary to allow ANTARES to supply AFS functionality to the Rubin data rights community.

\section{Rubin Requirements for Alert Filtering} \label{sec:requirements}

\section{The ANTARES Alert Broker} \label{sec:antares}

\section{Implementing the Rubin AFS with ANTARES} \label{sec:implementation}

\appendix
% Include all the relevant bib files.
% https://lsst-texmf.lsst.io/lsstdoc.html#bibliographies
\section{References} \label{sec:bib}
\renewcommand{\refname}{} % Suppress default Bibliography section
\bibliography{local,lsst,lsst-dm,refs_ads,refs,books}

% Make sure lsst-texmf/bin/generateAcronyms.py is in your path
\section{Acronyms} \label{sec:acronyms}
\input{acronyms.tex}
% If you want glossary uncomment below -- comment out the two lines above
%\printglossaries





\end{document}
